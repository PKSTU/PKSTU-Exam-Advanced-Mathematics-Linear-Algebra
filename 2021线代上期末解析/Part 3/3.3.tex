设 $\overline{\vb A}$ 为 $\vb A$ 的增广矩阵, 则
\[
	\overline{\vb A}=\left[\begin{array}{c:c}\vb{A}&\vb{b}\end{array}\right]=\left[\begin{array}{ccccc:c}
	1 & -2 & 1 & 1 & -5  & 3 \\
2 & -4 & 3 & 1 & -11 & 6 \\
1 & -2 & 2 & 3 & -12 & 6 \\
3 & -6 & 3 & 2 & -13 & t\end{array}\right]\to\left[\begin{array}{ccccc:c}
1 & -2 & 0 & 0 & 0  & 9-t \\
0 & 0  & 1 & 0 & -3 & 9-t \\
0 & 0  & 0 & 1 & -2 & 9-t \\
0 & 0  & 0 & 0 & 0  & t-8
\end{array}\right]
\]
要使线性方程组组有解, 则 $r(\vb A)=r(\overline{\vb A})$, 此时 $t=8$. 
\[
	\overline{\vb A}=\left[\begin{array}{ccccc:c}
1 & -2 & 0 & 0 & 0  & 1 \\
0 & 0  & 1 & 0 & -3 & 1 \\
0 & 0  & 0 & 1 & -2 & 1 \\
0 & 0  & 0 & 0 & 0  & 0
\end{array}\right]
\]
由此得方程组由自由未知量表示的通解为
\[
	\left\{
\begin{aligned}
x_1=2x_2+1\\
x_3=3x_5+1\\
x_4=2x_5+1
\end{aligned}
	\right.
\]
于是得方程组的结构式通解为
\[
	\vb{x}=\begin{pmatrix}1\\0\\1\\1\\0\end{pmatrix}+c_1\begin{pmatrix}2\\1\\0\\0\\0\end{pmatrix}+c_2\begin{pmatrix}0\\0\\3\\2\\1\end{pmatrix}
\]
其中 $c_1,c_2$ 为任意常数.