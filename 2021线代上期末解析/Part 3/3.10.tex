

\begin{enumerate}
    \item 由于\[\begin{aligned}
        \boldsymbol{A}^2  & = (\boldsymbol{E} - \boldsymbol{\alpha\alpha}\t) (\boldsymbol{E} - \boldsymbol{\alpha\alpha}\t) = \boldsymbol{E} - 2 \boldsymbol{\alpha\alpha}\t + \boldsymbol{\alpha\alpha}\t \boldsymbol{\alpha\alpha}\t  \\ & = \boldsymbol{E} - 2\boldsymbol{\alpha\alpha}\t + (\boldsymbol{\alpha}\t \boldsymbol{\alpha})\boldsymbol{\alpha\alpha}\t = \boldsymbol{E} - (2 - \boldsymbol{\alpha}\t \boldsymbol{\alpha})\boldsymbol{\alpha\alpha}\t
    \end{aligned}\]
    其中\(\boldsymbol{\alpha}\t \boldsymbol{\alpha}\)可以提出来是因为\(\boldsymbol{\alpha}\t \boldsymbol{\alpha}\)是一个数。故\[\boldsymbol{A}^2 = \boldsymbol{A} \qquad \Longleftrightarrow\qquad \boldsymbol{E} - \boldsymbol{\alpha\alpha}\t = \boldsymbol{E} - (2 - \boldsymbol{\alpha}\t \alpha) \boldsymbol{\alpha\alpha}\t\qquad\Longleftrightarrow\qquad \boldsymbol{\alpha}\t \boldsymbol{\alpha} = 1\]

    \item 由1知,\[\boldsymbol{\alpha}\t \boldsymbol{\alpha}\qquad\Longleftrightarrow\qquad\boldsymbol{A}^2 = \boldsymbol{A}\]我们断言:\underline{\(\boldsymbol{A}\)是降秩矩阵}。若不然,反设\(\boldsymbol{A}\)满秩,从而可逆。那么\[\begin{aligned}
        \boldsymbol{A}^2  = \boldsymbol{A} \qquad & \Longrightarrow \qquad \boldsymbol{A}^2\boldsymbol{A}^{-1}  = \boldsymbol{AA}^{-1}\qquad \Longrightarrow \qquad \boldsymbol{A} = \boldsymbol{E} \\ &\Longrightarrow\qquad \boldsymbol{A} = \boldsymbol{E} - \boldsymbol{\alpha\alpha}\t = \boldsymbol{E}\qquad \Longrightarrow\qquad\boldsymbol{\alpha\alpha}\t = 0
    \end{aligned}
        \]
        这与\(\boldsymbol{\alpha\alpha}\t = 1\)矛盾,故\(\boldsymbol{A}\)是降秩矩阵。
\end{enumerate}