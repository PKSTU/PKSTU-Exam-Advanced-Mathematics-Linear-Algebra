\usepackage{tcolorbox,listings,hlist,enumerate,ulem,ifthen,ctex}
\usepackage[left=1cm,right=1cm,top=1.5cm,bottom=1.5cm]{geometry}
\usepackage{amscd,amssymb,amsfonts,amsbsy,amsmath,verbatim,color, mathrsfs,yhmath,tkz-euclide,chemfig,siunitx,circuitikz}
\usepackage[version=4]{mhchem}
\usepackage{asymptote}\usepackage{framed}
\usepackage{lastpage}
\usepackage{geometry} %调整页面边框
\geometry{a4paper,scale=0.8}%调整到80%
\usepackage{graphicx}
\usepackage{tikz}
\usepackage{multirow}%纵向合并表格
\usetikzlibrary{shapes.geometric,through,decorations.pathmorphing,arrows.meta,quotes,mindmap,shapes.symbols,shapes.arrows,automata,angles,3d,trees,shadows,automata,arrows,shapes.callouts,patterns,through,hobby}
\usetikzlibrary{intersections}%应用此 library 找到两条曲线的交点
\usetikzlibrary{calc} % 引入计算支持
\usepgflibrary{fpu}%调用 fpu 程序库,在计算线段与曲线、曲线与曲线的交点时使用这个库。
\newcommand{\RNum}[1]{\uppercase\expandafter{\romannumeral #1\relax}}%输入罗马数字
\usepackage{pgfplots}
\pgfplotsset{compat=1.17}
\usetikzlibrary{arrows}
\pagestyle{empty}
\usepackage{mathrsfs}
\usepackage{diagbox}%制作带斜线表头的宏包
\usepackage{pifont}%输入带圈数字
\usepackage{verbatim}%区间注释宏包
\usepackage{setspace}
\usepackage{fancyhdr}
\usepackage{tasks}
\settasks{label=\Alph*.,
	label-offset={0.5em},
	label-align=left,
	column-sep={2pt},
	item-indent={1.3em},before-skip={-0.7em},after-skip={-0.7em}}
\usepackage{ifthen}
\usepackage{array}

%---------- 选择题和填空题设置  --------------
  %选择题的4个选项,使用一个命令根据选项内容长度自动排版
\newlength{\lab}
\newlength{\lb}
\newlength{\lc}
\newlength{\ld}
\newlength{\lhalf}
\newlength{\lquarter}
\newlength{\lmax}
\newcommand{\xx}[4]{%%%%%%%%%
	\settowidth{\lab}{A.~#1~~~}
	\settowidth{\lb}{B.~#2~~~}
	\settowidth{\lc}{C.~#3~~~}
	\settowidth{\ld}{D.~#4~~~}
	\ifthenelse{\lengthtest{\lab > \lb}}  {\setlength{\lmax}{\lab}}  {\setlength{\lmax}{\lb}}
	\ifthenelse{\lengthtest{\lmax < \lc}}  {\setlength{\lmax}{\lc}}  {}
	\ifthenelse{\lengthtest{\lmax < \ld}}  {\setlength{\lmax}{\ld}}  {}
	\setlength{\lhalf}{0.5\linewidth}
	\setlength{\lquarter}{0.25\linewidth}
	\ifthenelse{\lengthtest{\lmax > \lhalf}}
	{%
		\begin{hlist}[pre skip=0pt,item skip=0pt,,item offset={1.5em}, label=\Alpha {hlisti}.,pre label={}]1
			\hitem #1
			\hitem #2
			\hitem #3
			\hitem #4
		\end{hlist}
	}  %%%
	{%%
		\ifthenelse{\lengthtest{\lmax > \lquarter}} % 
		{%
			\begin{hlist}[pre skip=0pt,item skip=0pt,item offset={1.5em}, label=\Alpha {hlisti}.,pre label={}]2
				\hitem #1
				\hitem #2
				\hitem #3
				\hitem #4
			\end{hlist}
		}
		{%
			\begin{hlist}[\parskip=0pt,pre skip=0pt,item skip=0pt,item offset={1.5em}, label=\Alpha {hlisti}.,pre label={}]4
				\hitem #1
				\hitem #2
				\hitem #3
				\hitem #4
			\end{hlist}
}}}

\newcommand{\tk}[2][2]{\uline{\makebox[#1cm][c]{%
				\ifanswer
				\textcolor{red}{#2}%
				\else
				\phantom{#2}%
				\fi}}}%填空答案
\newcommand{\kh}[1]{\hfill(\  {{%
			\ifanswer
			{\makebox[0.4cm][c]{\textcolor{red}{#1}}}%
			\else
			\phantom{\makebox[0.4cm][c]{#1} }%
			\fi}\  )}}%选择答案
	
\newif\ifanswer
\newcommand{\answer}[1]{\ifthenelse{\isodd{#1}}{\answertrue}{}}
\newcommand{\ty}[2]{\textcolor{blue}{\  {{%
				\ifty
				{%
				#1}%
				\else
				%\vspace*{8cm}%这里是空出的答题区域。也可以在exam-2022.tex中的题目下方插入合适的空白区域大小
				\fi}\  }}}%简答题答案/答案解析				
\newif\ifty
\newcommand{\tiyuan}[1]{\ifthenelse{\isodd{#1}}{\tytrue}{}}

\newcommand{\who}[1]{{{ %
			\ifwho
			{\hfill(\ \makebox[2.0cm][c]{\textcolor{blue}{#1}}\  )}%
			\else
			\phantom{\makebox[2.0cm][c]{#1} }%
			\fi}}}%(供题人:xxx)			
\newif\ifwho
\newcommand{\zuozhe}[1]{\ifthenelse{\isodd{#1}}{\whotrue}{}}